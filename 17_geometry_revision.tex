\documentclass{article}
\usepackage{amsmath}
\usepackage{amsfonts}
\usepackage{geometry}
\usepackage[utf8]{inputenc}
\usepackage{enumitem}
\usepackage{physics}
\setlength{\parindent}{0pt}
\setlength{\parskip}{1em}
\relpenalty=10000
\binoppenalty=10000

\begin{document}
\section{Geometry revision}
\begin{enumerate}
  \item % http://www.pdmi.ras.ru/EIMI/2018/Baltic_way/bw18coord.pdf
  The points $A$, $B$, $C$, $D$ lie, in this order, on a circle $\omega$, where $AD$ is a diameter of $\omega$. Furthermore, $AB = BC = a$ and $CD = c$ for some relatively prime integers $a$ and $c$. Show that if the diameter $d$ of $\omega$ is also an integer, then either $d$ or $2d$ is a perfect square.

  \item %IMO SL2018
  Let $ABCDE$ be a convex pentagon such that $AB = BC = CD$, $\angle EAB =  \angle BCD$, and $\angle EDC = \angle CBA$. Prove that the perpendiular line from $E$ to $BC$ and the line segments $AC$ and $BD$ are concurrent.

  \item % http://www.math.olympiaadid.ut.ee/arhiiv/varia/bwtr/bw18tr/bw18tren.pdf
  The heights of triangle $ABC$ for triangle $A_1B_1C_1$. The heights of triangle $A_1B_1C_1$ form triangle $A_2B_2C_2$. Prove that $ABC \sim A_2B_2C_2$

  \item % http://www.math.olympiaadid.ut.ee/arhiiv/varia/bwtr/bw18tr/bw18tren.pdf
  Vertex $A$ of square $ABCD$ is symmetric to the midpoint of side $CD$ with respect to line $l$. Find the ratio of the areas of the two quadrilaterals on either side of line $l$ which make up the square.

  \item % http://www.math.olympiaadid.ut.ee/arhiiv/varia/bwtr/bw18tr/bw18tren.pdf
  Acute triangle $ABC$ has circumcircle $\omega$. The tangents of $\omega$ at points $B$ and $C$ intesect at $P$. $D$ and $E$ are the projections of $P$ to lines $AB$ and $AC$ respectively. Prove that the orthocentre of triangle $ADE$ coincides with the midpoint of line $BC$.

  \item % http://www.pdmi.ras.ru/EIMI/2018/Baltic_way/bw18coord.pdf
  The bisector of the $\angle A$ of a triangle $ABC$ intersects $BC$ in a point $D$ and intersects the circumcircle of the triangle $ABC$ in a point $E$. Let $K, L, M$ and $N$ be the midpoints of the segments $AB, BD, CD$ and $AC$, respectively. Let $P$ be the circumcenter of the triangle $EKL$, and $Q$ be the circumcenter of the triangle $EMN$. Prove that $\angle PEQ = \angle BAC$.


\end{enumerate}

\end{document}
