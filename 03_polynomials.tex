\documentclass{article}
\usepackage{amsmath}
\usepackage{amssymb}
\usepackage{eqnarray}
\usepackage[utf8]{inputenc}

\setlength{\parindent}{0pt}
\setlength{\parskip}{1em}

\newcommand{\K}{\mathbb{K}}
\newcommand{\N}{\mathbb{N}}
\newcommand{\Z}{\mathbb{Z}}
\newcommand{\R}{\mathbb{R}}
\newcommand{\C}{\mathbb{C}}
\newcommand{\Q}{\mathbb{Q}}
\newcommand{\F}{\mathcal{F}}
\newcommand{\G}{\mathcal{G}}

\begin{document}

\section{Polynomials}

\textbf{Definition 1.}  Polynomial function $P \colon \mathbb{R} \to \mathbb{R}$ can be presented in the form of
$$P(x)=a_nx^n+a_{n-1}x^{n-1}+\dots+a_1x+a_0$$
where $a_0,\dots,a_n$ are real numbers called the polynomial coefficients. Largest $n$ for which $a_n\ne 0$ is called the degree of polynomial.


\textbf{Theorem 2 (Bezout's theorem).}
A polynomial $P(x)$ is divisible by the binomial $(x-a)$ if and only if $P(a)=0$.

\textbf{Theorem 3 (The fundamental theorem of algebra).}
Every non-constant polynomial has a complex root.

\textbf{Theorem 4 (The rational root theorem).} If $x = p/q$ is a rational zero
of a polynomial $P(x) = a_nx^n +\hdots +a_0$ with integer coefficients and $(p, q)=1$,
then $p | a_0$ and $q | a_n$.

\textbf{Theorem 5 (Vieta's formulae).} If the solutions polynomial of degree $n$ are $x_1,x_2,\dots,x_n$ and  $a_n=1$, then the following holds:
\begin{eqnarray*}
x_1+x_2+\ldots+x_n &=& -a_{n-1},\\
x_1x_2+x_1x_3+\ldots+x_{n-1}x_{n} &=& \hphantom{-} a_{n-2}, \\
x_1x_2x_3+x_1x_2x_4+\ldots+x_{n-2}x_{n-1}x_n &=& -a_{n-3},\\
\ldots \\
x_1x_2\ldots x_n &=& (-1)^n a_0.
\end{eqnarray*}

\begin{enumerate}

\item
Find the roots of polynomial $P(x)=x^5-x^4-13x^3+x^2+12x$.

\item % Talvine lahtine võistlus 2010, noorem rühm
Find the value of $x^5+2x^2-4x+2010$ given that $x^3+2x+2=0$.

\item
How many points do we need to uniquely define a polynomial of degree $n$?

\item
The roots for polynomial $P_2(x)=ax^2+bx+c$ are $x_1$ and $x_2$. Find the coefficients for third order polynomial which has roots $x_1^2$, $x_2^2$ and $x_1x_2$.

\item % PSS 24
In $x^3+px^2+qx+r$ one root is the sum of the two others. Find the relationship between $p$, $q$ and $r$. 

\item % http://eqworld.ipmnet.ru/en/solutions/ae/ae0106.pdf
Find the roots of the polynomial $P(x)=ax^4+bx^3+cx^2+bx+a$


\item %PSS 31
Polynomial with integer coefficients $ax^3+bx^2+cx+d$ has $ad$ odd and $bc$ even. Show that at least one zero of the polynomial is irrational.

\item % PRoblem solving strategies 72
Polynomial of degree $n$ with non-negative coefficients and leading coefficient $1$ has $n$ real roots. Prove that $P(2)\geq 3^n$. 


\end{enumerate}

\end{document}