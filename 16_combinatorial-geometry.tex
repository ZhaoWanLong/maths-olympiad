\documentclass{article}
\usepackage{amsmath}
\usepackage{amsfonts}
\usepackage{geometry}
\usepackage[utf8]{inputenc}
\usepackage{enumitem}
\usepackage{physics}
\setlength{\parindent}{0pt}
\setlength{\parskip}{1em}
\relpenalty=10000
\binoppenalty=10000

\begin{document}
\section*{Combinatorial geometry}
	\begin{enumerate}
		\item % PKV 10 2017
		There are $5$ points on the plane, no three of them lie on the same line. Prove that there exists $4$ of them which form a convex quadrilateral
		\item % LVS V 2017
		A finite number of points is chosen on the plane, no three of them lie on the same line. It is known that there exists a non-convex polygons with its vertices at given points. Prove that there exists a non-convex quadrilateral with its vertices at given points.
		\item % BT 2017
		Let $n \geq 3$ be an integer. Find the largest number of angles which can be greater than $180^\circ$ in an $n$-gon whose sides are all equal.
		\item % LPV 11 2016
		Every point on the sides of an equilateral triangle is coloured either red or blue. Is it always possible to find a right angle triangle with all its vertices having the same colour.
		\item % PKV 11 2016
		Prove that there are more than $30000$ points with integral coordinates which lie within a circle of radius $100$.
		\item % VV 2013
		There are $n$ points on the plane. Starting from one of those points, in each step we move to the second closest point. After $n$ steps we have visited all the points and returned to the original point. Find all possible values for $n$.
		\item % http://www.math.olympiaadid.ut.ee/arhiiv/valik/vv2018/tvv2018.pdf ül5
		Let $k$ be a positive integer. Find all positive integers $n$ for which it is possible to choose $n$ points on the sides of a triangle (different from its vertices) and connect some of them with a line such that
		\begin{enumerate}
			\item There is at least $1$ point on each side
			\item For each pair of points $X$ and $Y$ which are on different sides of the triangle, there exists exactly $k$ points on the third side which are all connected to both $X$ and $Y$, and exactly $k$ points which are all connected to neither of $X$ or $Y$.
		\end{enumerate}

  \end{enumerate}
\end{document}
