\documentclass{article}
\usepackage{amsmath}
\usepackage{amsfonts}
\usepackage{geometry}
\usepackage[utf8]{inputenc}
\usepackage{enumitem}
\usepackage{physics}
\setlength{\parindent}{0pt}
\setlength{\parskip}{1em}
\relpenalty=10000
\binoppenalty=10000
\renewcommand{\thesubsection}{\arabic{subsection}}

\begin{document}
\section*{Functional equations}

\subsection{Injective and surjective functions}
	\begin{itemize}
		\item 
		If $f(x)=f(y) \Rightarrow x=y$, then $f$ is injective
		\item 
		If for each element $y$ in function codomain, there exists $x$ for which $f(x)=y$, then $f$ is surjective.
		\item
		If $f$ is both injective and surjective then $f$ is bijective.
	\end{itemize}

	\subsubsection*{Problems}
		\begin{enumerate}
			\item
			Let $f: X \to Y$ and $g: Y \to X$ and $g(f(x))=x$. Prove that $f$ is injective and $g$ is surjective.
			\item 
			Prove that for any function $f: X \to Y$, there exists a set $Z$ and functions $g: X \to Z$ and $h: Z\to Y$, such that $g$ is injective and $h$ is surjective.
			\item 
			Find all strictly monotonic functions $f: \mathbb{R} \to \mathbb{R}$ which satisfy 
			$$f(x + f(y)) = f(x) + y$$
		\end{enumerate}

	\subsection{Cauchy functional equations}
		\begin{itemize}
			\item
			Find all functions $f: \mathbb{Q} \to \mathbb{Q}$ for which $f(x)+f(y)=f(x+y)$.
			\item
			Find all functions $f: \mathbb{Q^+} \to \mathbb{Q^+}$ for which $f(x)f(y)=f(xy)$.
			\item
			Find all functions $f: \mathbb{Q^+} \to \mathbb{Q}$ for which $f(x)+f(y)=f(xy)$.
			\item
			Find all functions $f: \mathbb{Q} \to \mathbb{Q^+}$ for which $f(x)f(y)=f(x+y)$.
			\item 
			If the preceding functional equalities satisfy any one of the following conditions, find the solutions for functions defined for real numbers in the respective ranges.
			\begin{itemize}
				\item The function is continuous at one point,
				\item The function is monotonic on any interval,
				\item The function is bounded on any interval.
			\end{itemize}
		\end{itemize}
		\subsubsection*{Problems}
			\begin{enumerate}[resume]
				\item Find all functions $f: \mathbb{Q} \to \mathbb{Q}$ for which
				$$f(x+y) + f(x-y) = 2f(x) + 2f(y) $$
				%\item 
				%Find all functions $f: \mathbb{R} \to \mathbb{R}$ which satisfy $f(x)=f(x/2)$
			\end{enumerate}

	\subsection{Recurrence relations}
		\begin{itemize}
			\item 
			A recurrence relation is a relation that determines the elements of a sequence $x_n, n \in \mathbb{N_0}$, as a function of previous elements. A recurrence relation of the form
			$$(\forall n \geq k) \hspace{2em} x_n + a_1x_n-1 + \hdots + a_kx_{n-k} = 0$$
			for constants $a_1, \hdots, a_k$ is called a linear homogeneous recurrence relation of order $k$.
			\item
			We define the characteristic polynomial of the relation as 
			$$P(x) = x^k + a_1x^{k-1} + \hdots + a_k$$
			\item
			Let $P(x)$	factorize as 
			$$P(x) = (x-\alpha_1)^{k_1}(x-\alpha_2)^{k_2} \hdots (x-\alpha_r)^{k_r}$$
			where $\alpha_1, \hdots , \alpha_r$ are distinct complex numbers and $k_1, \hdots, k_r$ are positive integers. 
			\item 
			The general solution of this recurrence relation is in this case given by
			$$x_n = p_1(n)\alpha_1^n + p_2(n)\alpha_2^n + \hdots + p_r(n)\alpha_r^n$$
			where $p_i$ is a polynomial of degree less than $k_i$.
			\item
			In particular, if $P(x)$ has $k$
			distinct roots, then all $p_i$ are constant.
			\item 
			If $x_0, \hdots, x_{k-1}$ are set, then the coefficients of the polynomials are uniquely determined.
		\end{itemize}
			
			\subsubsection*{Problems}
			\begin{enumerate}[resume]
				\item 
				Find the closed form expression for $n$-th term of the Fibonacci sequence.
				\item
				Let $S_n$ denote the number of ternary sequences (consisting of $0$,$1$, and $2$s) of length $n$, such that they do not contain a substring of "10", "01", or "11". Find a closed form expression for $S_n$.
				\item
				Let $r$ be a real number, and let $x_n$ be a sequence such that $x_0 = 0, x_1 = 1$, and $x_{n+2} = rx_{n+1} - x_n$ for $n \ge 0$. For which values of $r$ does $x_1 + x_3 + \cdots + x_{2m-1} = x_m^2$ for all positive integers $m$?
				\item
				Let $a_{n}$, $b_{n}$, and $c_{n}$ be geometric sequences with different common ratios and let $a_{n}+b_{n}+c_{n}=d_{n}$ for all integers $n$. If $d_{1}=1$, $d_{2}=2$, $d_{3}=3$, $d_{4}=-7$, $d_{5}=13$, and $d_{6}=-16$, find $d_{7}$.
			\end{enumerate}	
\end{document}