\documentclass{article}
\usepackage{amsmath}
\usepackage{amsfonts}
\usepackage{geometry}
\usepackage[utf8]{inputenc}
\usepackage{enumitem}
\setlength{\parindent}{0pt}
\setlength{\parskip}{1em}

\begin{document}

\section{Functions in number theory}
\subsection{Rules}
\begin{enumerate}
\item
\textbf{Fermat's little theorem.} \\
For prime $p$ and integer $a$
$$a^p \equiv a \mod p$$

\item
\textbf{Wilson's theorem.} \\
$$(p-1)! \equiv -1 \mod p$$
if and only if when $p$ is prime number.

\item 
\textbf{Number of factors} \\
The number of positive factors of $n=p_1^{\alpha_1} \dots p_k^{a_k}$
$$d(n)= (\alpha_1+1)(\alpha_2+1)\dots (\alpha_k+1)$$	
\item 
\textbf{Sum of factors} \\
The sum of positive factors of $n=p_1^{\alpha_1} \dots p_k^{a_k}$

$$\sigma(n)= \frac{p_1^{\alpha_1+1}-1}{p_1-1} \dots  \frac{p_k^{\alpha_k+1}-1}{p_k-1}$$

\item 
\textbf{Euler's function} \\
Euler’s function or totient function $\varphi(n)$ is defined for $n=p_1^{\alpha_1} \dots p_k^{a_k}$ as the number
of positive integers less than $n$ and coprime to $n$. Then
$$\varphi(n) = n \left(1-\frac{1}{p_1}\right) \dots  \left(1-\frac{1}{p_k}\right)$$


\item 
\textbf{Euler's theorem}  (Generalisation of Fermat's theorem)\\
Let $n$ be a natural number and $a$ an integer such that $\gcd(a,n)=1$. Then
$$a^{\varphi(n)} \equiv 1 \mod n $$

\end{enumerate}

\subsection{Problems}
\begin{enumerate}

\item 
Find all primes $p$, for which the sum of all positive factors of $p^4$ is a perfect square.

\item 
Prove that for positive integer $n$ 
$$\sum_{d|n} \varphi (d) =n $$

\item 
Prove that for positive integers $a$ and $b$ 
$$ \varphi(ab) = \varphi (a)\varphi(b)\frac{\gcd(a,b)}{\varphi(\gcd(a,b))} $$

\item % https://artofproblemsolving.com/wiki/index.php?title=Fermat%27s_Little_Theorem#Problems
One of Euler's conjectures was disproved in the 1960s by three American mathematicians when they showed there was a positive integer such that $133^5+110^5+84^5+27^5=n^5$. Find the value of ${n}$. 

\item  %p46 http://s3.amazonaws.com/aops-cdn.artofproblemsolving.com/resources/articles/olympiad-number-theory.pdf
How many prime numbers $p$ are there, such that $29^p+1$ is a multiple of $p$?

\item 
Prove that if $\gcd(a,n)=1$, then
$$a^b \equiv a^{b \mod \varphi(n)}  \mod n$$

\item % http://s3.amazonaws.com/aops-cdn.artofproblemsolving.com/resources/articles/olympiad-number-theory.pdf
Find the last three digits of $2008^{2007^{2006^{\dots^{^{}2^{1}}}}}$.

\item 
Prove that there exists no positive integer for which $n! + 19^{93}$ is a perfect square.

\item 
Find all pairs of positive prime numbers $(p_1,p_2)$ for which the equation $$\phi(n^2) = n + p_1p_2$$
has a solution for $n$ in positive integers. 

\item % http://www.math.olympiaadid.ut.ee/eng/archive/bw/bw05sol.pdf

Let $p$ be a prime number and let $n$ be a positive integer. Let $q$ be a positive divisor of $(n+1)^p-n^p$. Show that $q-1$ is divisible by $p$.

\end{enumerate}


\end{document}