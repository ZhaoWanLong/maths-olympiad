\documentclass{article}
\usepackage{amsmath}
\usepackage{amsfonts}
\usepackage{geometry}
\usepackage[utf8]{inputenc}
\usepackage{enumitem}
\usepackage{physics}
\setlength{\parindent}{0pt}
\setlength{\parskip}{1em}
\relpenalty=10000
\binoppenalty=10000

\begin{document}

\section*{Graph theory}
\subsection*{Theory}
\begin{enumerate}
	\item 
	Graph $G(V,E)$ is set of \emph{vertices} $V$ and \emph{edges} $E$ (pairs of vertices). If there is an edge $(x,y) \in E$, then $x$ and $y$ are said to be \emph{connected}.
	\item 
	A \emph{proper graph} is a graph with no more than one edge between each pair of vertices and in which no vertex is connected to itself.
	\item 
	Proper graph in which each pair of vertices is connected is called a \emph{complete graph}.
	\item
	A \emph{complement} graph $\overline{G}$ has vertices equal to vertices of $G$ and its vertices $x$ and $y$ are connected if and only if $x$ and $y$ are not connected in $G$.
	
	\item 
	An \emph{oriented graph} is one in which the pairs in the graph are ordered.
	\item
	The vertices of a \emph{k-partite graph} can be partitioned into $k$ non-empty disjoint sets in such way that there are no connections between vertices in each set.
	\item 
	The \emph{degree} of a vertex $x$ is the number of times $x$ is the endpoint of an edge. Then
	$$\sum_{x \in V} d(x) = 2 \abs{E} $$
	\item 
	A \emph{trajectory (or path)} is a sequence of vertices for which subsequent vertices are connected.
	\item 
	A \emph{circuit} is a path that ends and starts in the same vertex.
	\item 
	A \emph{cycle} is a circuit in which no vertex appears more than once (except the initial/final vertex).
	\item
	In a \emph{connected graph} there exists a path between any two points of the graph.
	\item 
	\emph{Tree} is a connected graph with no circuits. A connected graph with $n$ vertices is a tree if and only if it has $n-1$ edges.
	\item 
	\emph{Euler path} is a path in which every edge of the graph appears exactly once. Likewise \emph{Euler circuit} is a circuit in which every edge appears exactly once.
	\item 
	If each vertex in a connected graph has even degree, then the graph contains an Euler circuit.
	\item 
	If a connected graph has exactly two vertices with odd degree, it contains an Euler path.
	\item 
	A \emph{Hamilton circuit} is a circuit in which each vertex appears exactly once.
	\item 
	A \emph{planar graph} can be embedded in a plane with edges corresponding to non-intersecting lines (not necessarily straight). A planar graph with $n$ vertices has at most $3n-6$ edges.
	
\end{enumerate}

\newpage
\subsection*{Problems}
\begin{enumerate}
	\item
	Prove that at least one of $G$ and $\overline{G}$ is connected.
	
	\item 
	\emph{(Dirac's theorem).} Prove that a graph with $n$ vertices contains a Hamilton cycle if the degree of each vertex is at least $n/2$. 	
	
	\item 
	\emph{(Euler's formula).}  A convex polyhedron has $E$ edges, $F$ faces and $V$ vertices. Prove that $E+2=F+V$.

	\item % http://www.math.olympiaadid.ut.ee/arhiiv/baltitee/bt1997/bw97est.pdf
	In a forest each of n animals ($n \leq 3$) lives in its own cave, and there is exactly one separate path
	between any two of these caves. Before the election for King of the Forest some of the animals
	make an election campaign. Each campaign-making animal visits each of the other caves exactly
	once, uses only the paths for moving from cave to cave, never turns from one path to another
	between the caves and returns to its own cave in the end of its campaign. It is also known that
	no path between two caves is used by more than one campaign-making animal.
	\begin{enumerate}
		\item
		Prove that for any prime $n$, the maximum possible number of campaign-making animals is $\tfrac{n-1}{2}$
		\item 
		Find the maximum number of campaign-making animals for $n = 9$.	
	\end{enumerate}
	
	\item % https://mks.mff.cuni.cz/kalva/usa/usoln/usol991.html
	Certain squares of an $n \times n$ board are coloured black and the rest white. Every white square shares a side with a black square. Every pair of black squares can be joined by chain of black squares, so that consecutive members of the chain share a side. Show that there are at least $\tfrac{n^2-2}{3}$ black squares.
	
	
	\item %https://cms.math.ca/Competitions/OMC/archive/sol2006.pdf
	Consider a round-robin tournament with $2n + 1$ teams, where each team plays each other team exactly once. We say
	that three teams $X$, $Y$ and $Z$, form a \emph{cycle triplet} if $X$ beats $Y$ , $Y$ beats $Z$, and $Z$ beats $X$. There are no ties.
	\begin{enumerate}
		\item 
		Determine the minimum number of cycle triplets possible.
		\item 
		Determine the maximum number of cycle triplets possible.
	\end{enumerate}
	 
	\item % https://artofproblemsolving.com/wiki/index.php?title=2001_IMO_Shortlist_Problems/C3
	Define a $k$-clique to be a set of $k$ people such that every pair of them are acquainted with each other. At a certain party, every pair of 3-cliques has at least one person in common, and there are no 5-cliques. Prove that there are two or fewer people at the party whose departure leaves no 3-clique remaining.	
	
	
\end{enumerate}


\end{document}